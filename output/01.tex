\section{昨今のDIY事情}\label{ux6628ux4ecaux306ediyux4e8bux60c5}

DIY(Do It Yourself)でできる範囲が広がってきています。
一昔前であれば、DIY≒日曜大工のイメージが強かったかと思います。
しかし最近は以下のように、「メカ」「エレキ」「ファームウェア」の3要素で構成される「モノづくり」を個人ができる環境が整いつつあります。

\begin{itemize}
\tightlist
\item
  メカ \par
     3Dプリンタ、レーザーカッター、無料3D CAD(Fusion360、Cleo
  Elements、等)
\item
  エレキ \par
     ArduinoやRaspberry Piなどのマイコンボードの搭乗
\item
  ファームウェア \par
    オープンソースソフトウェアの充実
\item
  メイカースペースの充実 \par
    DMM. MAKE、Tech Shop Japan、FabLab、Maker's
  Base、など加工機を使用できる環境が増えてきました
\item
  部品調達 \par
     インターネットの発展に伴い、個人で低価格・小ロットの部品試作ができたり、既成部品を安価に調達することが可能となった
\item
  資本金調達\par 
     Kick
  Starterを始めとしたクラウドファウンディングにより資本が調達しやすくなった
\end{itemize}

個人がモノを作れるようになったこの一連の流れは「Maker
Movement」と呼ばれています。
私が学生だった頃は一般家庭で使えるような3Dプリンタが家電量販店で売られる時代が来るとは思いもしませんでした。
良い時代になったものです。

\section{本誌の中身}\label{ux672cux8a8cux306eux4e2dux8eab}

前述した「Maker
Movement」が生まれた要因の一つが3Dプリンタの登場であり、本誌がテーマとしているところです。
本誌では3Dプリンタの特徴や使いこなし方法を解説していきます。

モノを作るときにはツールの特徴を考えて設計を行う必要がありますので
家庭用3Dプリンタの特徴を説明した上で設計のコツや注意することを述べていきます。

具体的な使用例としては、筆者が参加している「かわさきロボット競技大会」の機体製作を取り上げます。
コンセプト設計から具体的な製作まで、一連の製作の流れを示すとともに、3Dプリンタで部品を作る際に苦労したこと・工夫したことを記載していきます。

尚、部品の詳細形状に関しては定性的な内容が多くを占めていますのでご注意ください。
使用する3Dプリンタによって積層方法が異なるため一般化が困難なのと、単純に時間が取れなかった、というのもあります。

\section{本誌の対象者}\label{ux672cux8a8cux306eux5bfeux8c61ux8005}

本誌の対象ですが、「3Dプリンタを初めて使う機械系専攻の学生」をイメージしています。
章ごとの内容は下記の通りです。
第3、4章では私が個人的に参加しているロボットコンテスト用の機体をベースに説明していきます。

\begin{itemize}
\tightlist
\item
  第2章\ldots{}\ldots{}3Dプリンタの解説・選定のポイント
\item
  第3章\ldots{}\ldots{}3Dプリンタの特徴を踏まえたロボット設計
\item
  第4章\ldots{}\ldots{}3Dプリンタを用いたロボット製作
\end{itemize}
